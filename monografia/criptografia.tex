\chapter{Criptografia} % (fold)
\label{cha:criptografia}
A Criptografia tem como objeto de estudo a investigação de informações deliberadamente ocultadas. É considerada por vários autores como ramo da Matemática e da Ciência da Informação. Já o ciframento é definido como o processo que transforma uma informação objetivando ocultá-la. \par

A ocultação de informações tem objetivos práticos bastante óbvios, especialmente quando se analisa o ambiente militar. Códigos de lançamento de armas, posições de tropas, informações de movimentação de suprimentos e munição, comunicação entre os diversos comandantes, são típicas informações às quais não se deseja que o inimigo tenha acesso, e, portanto, candidatas a proteção. Outros tipos de informação tendem a ser protegidas por seus detentores da mesma forma: receitas de produtos, dados financeiros de empresas e pessoas, segredos industriais, entre outros. \par

Diversas técnicas foram desenvolvidas para proteger essas informações (e para desvendá-las) ao longo da história. Desde mensageiros que levavam mensagens em finos papiros imersos em vinagre que destruiria a mensagem a não ser que se soubesse uma senha secreta, ou que levavam as informações dentro de seu próprio corpo até os modernos programas de computadores que cifram imensos textos em frações de segundo. O ciframento será o método a ser estudado neste texto por ser a base do processo de assinatura digital, 



\section{História da Criptografia} % (fold) 
\label{sub:historia_da_criptografia}

As origens da Criptografia remontam ao Antigo Egito aonde certos hierogrifos que fugiam aos padrões da época eram inscritos nas tumbas de seus faraós, provavelmente, pensa-se, para conferir ares místicos às informações lá representadas. Pouco mais tarde, na Mesopotâmia, algumas tábuas foram confeccionadas com informações criptografadas. Tratavam-se de receitas que provavelmente tinham algum valor comercial à época. \par

Ainda na Antiguidade, os Romanos utilizaram-se da criptografia para comunicação entre o Imperador e seus generais. Uma das mais famosas cifras cripográficas vem desta época e é chamada de ``Cifra de César'', em um dos usos militares da Criptografia. A demanda militar por sigilo nas comunicações levou as tropas de diversas nações durante a história a recorrer à Criptografia para evitar que mensagens interceptadas fossem corretamente interpretadas pelo inimigo.\par

Até a Idade Média, grande parte das 

% section historia_da_criptografia (end)


% chapter cha:criptografia (end)