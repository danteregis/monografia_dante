\chapter{Hipóteses}
\begin{itemize}
    \item O arquivo digital não assinado eletrônicamente não possui os pré-requisitos de um documento (imutabilidade, identificação e ), e não pode, portanto, ser considerado como manifestação válida da parte em um processo judicial.
    \item As partes têm o direito de exigir ratificação escrita, ou digitalmente assinada, do conteúdo das peças das partes adversas e também o de contestar o conteúdo das peças por elas produzidas quando não assinadas digitalmente por falha técnica ou inépcia do sistema informatizado do Poder Judiciário.
    \item Os processos judiciais exclusivamente virtuais, sem contra-parte escrita e devidamente autografada, não devem ser considerados válidos, em função da volatilidade de suas informações.
\end{itemize}
