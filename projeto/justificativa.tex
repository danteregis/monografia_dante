\chapter{Justificativa}

  Poucos problemas brasileiros atingem tanto consenso no tocante
  ao seu diagnóstico quanto a questão judiciária: a morosidade
  dos processos judiciais e a baixa eficácia de suas decisões
  atrapalham o desenvolvimento do país e provocam cautela por
  parte dos investidores ao propiciarem inadimplência, impunidade
  e comprometerem a credibilidade da democracia
  \cite{RelatorioCNJ2005}.\par

  O Judiciário, diante do clamor social por maior agilidade
  processual, procura formas de se reestruturar, ao tempo em que o
  Legislativo contribui com reformas na legislação processual
  tendente a racionalizar o processo e facilitar sua tramitação.
  \par

  Uma das soluções encontradas foi a virtualização do processo
  judicial. Procura-se, com o processo eletrônico, reduzir o
  tempo gasto na tramitação dos autos e contribuir para o fim da
  morosidade da Justiça.\par

  A solução atrai gestores dos diversos Tribunais, pois promete
  reduzir custos e aumentar a eficiência, racionalizando o uso
  dos recursos públicos. Iniciou-se, assim, uma enorme
  mobilização dos órgãos da Justiça no sentido de implantar seus
  sistemas informáticos de virtualização do processo
  judicial.\par

  É, portanto, de primeira necessidade o estudo deste tema, pois
  a virtualização, que se espalha por todo o país, importa a
  todos aqueles que, direta ou indiretamente, se relacionam com o
  Judiciário ou figuram como parte em qualquer um dos milhares de
  processos que lá tramitam.\par

  A doutrina ainda é inscipiente no que toca a validade jurídica
  dos documentos digitais para contratos e instrumentos
  particulares. Mas não se encontram estudos que analizem a
  validade do processo judicial virtual, que é formado, em
  essência, de vários documentos digitais produzidos pelas
  partes, pelo juiz e pelos diversos auxiliares da Justiça.\par

  Diferentemente do processo físico, composto por documentos
  escritos e assinados por seus autores, o processo virtual
  existe apenas como conjunto de dados armazenadas em computadores
  do órgão em que tramitam. E ainda de modo diverso de sua
  contraparte física, os arquivos digitais, em função da mídia
  em que são armazenados, têm por característica a volatilidade.
  De modo geral, não é possível determinar se o documento
  existente em um certo sistema de computadores é o mesmo que foi
  gravado inicialmente, ou mesmo quais as alterações por este
  sofridas desde a sua criação.\par

  Torna-se imperativa, portanto, a certeza quanto à fidelidade do
  conteúdo e tempestividade de documentos assim armazenados.\par



	Nos últimos anos, a Justiça Brasileira iniciou um processo visando a virtualização do processo judicial. Trata-se de uma mudança imposta pela notável demanda social por celeridade. Os Tribunais, então, visando atender a essa demanda, implementam sistemas informáticos de virtualização do processo judicial para agilizar o 

  Apoiados na constante demanda social pela celeridade, os
  Tribunais implementam seus sistemas informáticos de
  virtualização judicial e as decisões que transformam a vida da
  sociedade passam a existir apenas como conjuntos de anomalias
  magnéticas dentro de computadores dos diversos órgãos do Poder
  Judiciário.\par
    
    Diante da imperativa necessidade de certeza quanto a fidelidade do conteúdo do documentos armazenados dessa forma em relação a intenção do autor do documento e, especialmente, quanto à imutabilidade desses mesmos documentos, é que se faz imprescindível uma análise das implementações dos processos virtuais realizadas pelos Tribunais. \par
	
	Procura-se, com esse estudo, demonstrar aos juristas e, especialmente, à sociedade, que a fiscalização dos novos serviços digitais apresentados pelo Poder Judiciário é necessária para garantir a segurança jurídica. Se as implementações de sistemas processuais virtuais feitas de maneira inadequada podem levar a um estágio de insegurança inadmissível, a ponto de permitir questionar se o conteúdo de uma sentença publicada no Diário Oficial (também virtual, é bom que seja dito)  é o mesmo que foi determinado pelo juízo responsável, não se pode permitir que persitam sem que seja do conhecimento de todos essa possibilidade, pois cabe apenas à sociedade escolher se o anseio pela celeridade deve prevalescer sobre o fundamento mais elementar do Direito: a segurança .\par
