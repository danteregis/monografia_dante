\chapter{Justificativa}

  Poucos problemas brasileiros atingem tanto consenso no tocante
  ao seu diagnóstico quanto a questão judiciária: a morosidade
  dos processos judiciais e a baixa eficácia de suas decisões
  atrapalham o desenvolvimento do país e provocam cautela por
  parte dos investidores ao propiciarem inadimplência, impunidade
  e comprometerem a credibilidade da democracia
  \cite{RelatorioCNJ2005}.\par

  O Judiciário, diante do clamor social por maior agilidade
  processual, procura formas de se reestruturar, ao tempo em que o
  Legislativo contribui com reformas na legislação processual
  tendente a racionalizar o processo e facilitar sua tramitação.
  \par

  Uma das soluções encontradas foi a virtualização do processo
  judicial. Procura-se, com o processo eletrônico, reduzir o
  tempo gasto na tramitação dos autos e contribuir para o fim da
  morosidade da Justiça.\par

  A solução atrai gestores dos diversos Tribunais, pois promete
  reduzir custos e aumentar a eficiência, racionalizando o uso
  dos recursos públicos. Iniciou-se, assim, uma enorme
  mobilização dos órgãos da Justiça no sentido de implantar seus
  sistemas informáticos de virtualização do processo
  judicial.\par

  A doutrina ainda é inscipiente no que toca a validade jurídica
  dos documentos digitais para contratos e instrumentos
  particulares. E não se encontram estudos que analizem a
  validade do processo judicial virtual, que é formado, em
  essência, de vários documentos digitais produzidos pelas
  partes, pelo juiz e pelos diversos auxiliares e serventuários
  da Justiça.\par

  Diferentemente do processo físico, composto por documentos
  escritos e assinados por seus autores, o processo virtual
  existe apenas como conjunto de dados armazenadas em computadores
  do órgão em que tramitam. E ainda de modo diverso de sua
  contraparte física, os arquivos digitais, em função da mídia
  em que são armazenados, têm por característica a volatilidade:
  de modo geral, não é possível determinar se o documento
  existente em um certo sistema de computadores é o mesmo que foi
  gravado inicialmente, ou mesmo quais as alterações por este
  sofridas desde a sua criação \cite[p.554]{ComputerNetworks}.\par

  É, portanto, de primeira necessidade o estudo deste tema, pois
  a virtualização, que se espalha por todo o país, importa a
  todos aqueles que, direta ou indiretamente, se relacionam com o
  Judiciário ou figuram como parte em qualquer um dos inúmeros
  processos que lá tramitam. Afinal, como conviver com a
  incerteza de que a sentença publicada no sistema do Tribunal
  pode não ser a mesma que foi emitida pelo juíz? Como imaginar
  que a petição enviada por uma parte possa sofrer alterações sem
  que seja possível se dar conta disso?\par

  Para evitar essa insegurança, existem mecanismos adequados
  criados pelos cientistas da computação. Pode-se provar, por
  meio de equações e fórmulas matemáticas que o autor de
  determinado documento é realmente aquele que o subscreve e que
  a mensagem enviada não sofreu alterações no caminho ou depois
  de armazenada no Judiciário.\par

  Optar por não utilizar esses mecanismos é optar pela tolerância
  à fraude, pelo estímulo à torpeza. A segurança jurídica, tão
  protegida nas inúmeras sentenças dos mais diversos graus de
  jurisdição corre grave risco, provocado pela inobservância das
  mais básicas recomendações da Ciência da Computação em relação
  à integridade dos documentos eletrônicos. E o que é o processo
  judicial virtual senão um encadeamento organizado de documentos
  eletrônicos? A fragilidade do elemento põe em risco a
  integridade do conjunto, e o processo virtual padece do risco
  de fraude no exato momento em que são escritas essas linhas. Os
  mesmos tribunais que intentam proteger a segurança da sociedade
  a põem em risco em sua busca temerária pela celeridade.\par

  Será provado por esse estudo que é imperativa a adoção,
  pelos Tribunais, desses mecanismos de proteção à fraude
  digital. A crescente virtualização dos processos judiciais
  impede que sejam ignorados os problemas advindos das novas
  tecnologias. Não se deve permitir que o anseio pela celeridade
  prevalesça sobre o fundamento mais elementar do Direito: a
  segurança.
