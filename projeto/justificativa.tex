\chapter{Justificativa}

	Nos últimos anos, a Justiça Brasileira iniciou um movimento aparentemente inexorável em direção à virtualização do processo judicial. Apoiados na constante demanda social pela celeridade, os Tribunais implementam seus sistemas informáticos de virtualização judicial e as decisões que transformam a vida da sociedade passam a existir apenas como conjuntos de anomalias magnéticas dentro de computadores dos diversos órgãos do Poder Judiciário.\par
	
	Diante da imperativa necessidade de certeza quanto a fidelidade do conteúdo do documentos armazenados dessa forma em relação a intenção do autor do documento e, especialmente, quanto à imutabilidade desses mesmos documentos, é que se faz imprescindível uma análise das implementações dos processos virtuais realizadas pelos Tribunais. \par
	
	Procura-se, com esse estudo, demonstrar aos juristas e, especialmente, à sociedade, que a fiscalização dos novos serviços digitais apresentados pelo Poder Judiciário é necessária para garantir a segurança jurídica. Se as implementações de sistemas processuais virtuais feitas de maneira inadequada podem levar a um estágio de insegurança inadmissível, a ponto de permitir questionar se o conteúdo de uma sentença publicada no Diário Oficial (também virtual, é bom que seja dito)  é o mesmo que foi determinado pelo juízo responsável, não se pode permitir que persitam sem que seja do conhecimento de todos essa possibilidade, pois cabe apenas à sociedade escolher se o anseio pela celeridade deve prevalescer sobre o fundamento mais elementar do Direito: a segurança.\par
