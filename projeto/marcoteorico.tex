\chapter{Marco Teórico de Referência}
  \label{marcoteorico}

  Será adotado, por esse projeto, os conceitos de Assinatura
  Eletrônica, Autoridade Certificadora e de Certificado Digital
  encontrados em \citeonline{ComputerNetworks,
  SistemasOperacionaisModernos}, que, por seus trabalhos na área
  de redes, segurança e sistemas operacionais, é
  internacionalmente reconhecido como referência nesses temas,
  além de ter seus livros didáticos utilizados nos cursos de
  graduação de várias universidades ao redor do mundo. A
  Assinatura Digital é pelo autor definida como

  \begin{citacao}um sistema em que uma parte pode enviar uma
  mensagem a outra assinada de forma a assegurar três condições:
  \begin{enumerate} \item O receptor possa ter a garantia de que
  o remente é quem alega ser; \item O remetente não poderá,
  posteriormente, repudiar o conteúdo da mensagem; e \item O
  receptor não possa ter forjado a mensagem\end {enumerate}
  \cite{ComputerNetworks}\end{citacao}.\par

  Autoridade Certificadora, no conceito encontrado em
  \cite{ComputerNetworks}, é a organização que certifica que
  uma chave pública pertence a determinada pessoa física ou
  jurídica.\par

  O Certificado Digital, por sua vez, é definido por Tanenbaum
  \nocite{ComputerNetworks} como aquele emitido por uma
  Autoridade Certificadora e por ela assinado através de sua
  chave privada.\par

  
