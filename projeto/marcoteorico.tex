\chapter{Marco Teórico de Referência}
  \label{marcoteorico}

  Será adotado, por esse projeto, os conceitos de Assinatura
  Eletrônica, Autoridade Certificadora e de Certificado Digital
  encontrados em \citeonline{ComputerNetworks,
  SistemasOperacionaisModernos}, que, por seus trabalhos na área
  de redes, segurança e sistemas operacionais, é
  internacionalmente reconhecido como referência nesses temas,
  além de ter seus livros didáticos utilizados nos cursos de
  graduação de várias universidades ao redor do mundo. A
  Assinatura Digital é pelo autor definida como

  \begin{citacao}um sistema em que uma parte pode enviar uma
  mensagem a outra assinada de forma a assegurar três condições:
  \begin{enumerate} \item O receptor possa ter a garantia de que
  o remente é quem alega ser; \item O remetente não poderá,
  posteriormente, repudiar o conteúdo da mensagem; e \item O
  receptor não possa ter forjado a mensagem\end {enumerate}
  \cite{ComputerNetworks}\end{citacao}.\par

  Autoridade Certificadora, no conceito encontrado em
  \cite{ComputerNetworks}, é a organização que certifica que
  uma chave pública pertence a determinada pessoa física ou
  jurídica.\par

  O Certificado Digital, por sua vez, é definido por Tanenbaum
  \nocite{ComputerNetworks} como aquele emitido por uma
  Autoridade Certificadora e por ela assinado através de sua
  chave privada.\par

  Em relação ao Princípio da Instrumentalidade do Processo,
  adotaremos o conceito exposto por Cintra \emph{et. al}
  \nocite{TeoriaGeralDoProcesso}, por serem autores tradicionalmente
  utilizados nas faculdades de Direito brasileiras, e segundo os
  quais o processo judicial é instrumento dos desígnios do
  direito material, só merecendo serem cumpridas à risca as
  formalidades processuais na estrita medida em que sejam
  indispensáveis para a a consecução dos objetivos desejados
  \cite[p. 48]{TeoriaGeralDoProcesso}.

  Em um artigo sobre o tema da certificação digital e suas
  implicações no Direito, Boreggio e Nahasan
  \nocite{CertificadoDigital} deram a definição de Documento
  Eletrônico (ou, indiscriminadamente no contexto deste trabalho,
  Documento Digital) que será adotada por esse projeto,
  delimitando-o como representação da realidade em forma gráfica,
  textual, sonora ou qualquer outra admitida pela tecnologia, que
  tem como base qualquer meio capaz de garantir sua certeza e
  imutabilidade e possibilidade de atribuição a determinado autor
  \cite[p. 24]{CertificadoDigital}

  Por fim, para o Processo Judicial exclusivamente Virtual, que
  indiscriminadamente será também chamado de Processo Virtual e
  Processo Eletrônico, adotaremos a descrição legal, definindo-o
  como aquele que tramita exclusivamente através de meio
  eletrônico \cite{Lei11419}.
