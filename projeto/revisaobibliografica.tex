\chapter{Revisão de Bibliografia}

  A monografia de que trata este projeto tem por característica a
  análise de duas áreas do conhecimento humano: o Direito e a
  Ciência da Computação. Em virtude das peculiaridades do tema,
  serão nos autores nas construções teóricas dessas duas ciências
  que se irá buscar a solução para o problema proposto.\par

  Para se discutir a validade jurídica das peças processuais
  virtuais não assinadas digitalmente, é necessário,
  inicialmente, verificar o conceito de documento eletrônico e de
  assinatura digital.\par

  Via de regra, é possível determinar se um documento físico é
  autêntico pela assinatura do autor. Observa-se sua caligrafia,
  a tinta utilizada e eventuais marcas em alto-relevo. Um perito
  pode fazer análise grafotécnica para determinar se a caligrafia
  do texto é a mesma do autor ou se existem adulterações no texto
  original. Não é possível utilizar nenhum desses métodos para
  documentos digitais \cite{ComputerNetworks, CertificadoDigital}.\par

  Em relação a estes, a Ciência da Computação desenvolveu métodos
  de verificação da autenticidade e integridade da informação.
  Todos esses métodos estão relacionados de alguma forma à
  técnica da criptografia, conforme afirma
  \citeonline[p. 555]{ComputerNetworks}.\par

  O termo criptografia vem da expressão grega para "escrita
  secreta". A técnica é bastante antiga: um dos exemplos mais
  remotos de sua aplicação é o do exército romano na época de
  César, que deu origem à Cifra de César \cite[p.
  555]{ComputerNetworks}.\par

  Na utilização dessa técnica, um texto, produzido pelo emissor da
  mensagem, passa por um programa de computador que, através de
  processos matemáticos e de lógica, transforma-o em um conjunto de
  caracteres ininteligíveis, a que se chama de mensagem ou texto
  cifrado.  Esta é a mensagem que é transmitida ou gravada. Para
  a leitura dessa mensagem, é necessário que a mesma passe
  novamente por um processo lógico-matemático que transformará o
  texto cifrado novamente no original. A esse conjunto de
  operações pelas quais passa o texto, dá-se o nome de
  algorítimo \cite{ComputerNetworks}. Para evitar que apenas o
  conhecimento do algorítimo permita a um estranho ler a
  mensagem, muitos desses procedimentos lógicos passaram a adotar
  a "chave" criptográfica, sem a qual não se poderia ler a
  mensagem original. Para a transmissão do texto, remetente e
  receptor deveriam combinar entre si, de alguma forma, qual
  seria a chave utilizada.\par

  Vários são os tipos e implementações modernas de algorítimos
  criptográficos. Para o escopo deste trabalho, devemos nos ater aos 
  de chave assimétrica, mais precisamente no algorítimo RSA,
  utilizado no certificado digital.\par

  Este algorítimo surgiu em 1977, e utiliza o princípio de chaves
  públicas e privadas. Para enviar uma mensagem criptografada
  para o indivíduo \emph{A}, o indivíduo \emph{B} utilizará a
  chave pública de \emph{A}. Esta chave, como o nome esclarece, é
  de conhecimento amplo, e pode ser livremente divulgada. Para
  ler a mensagem, no entanto, não se utilizará a mesma chave que
  criptografou o texto, mas uma diferente, denominada chave
  privada. Somente \emph{A}, possuidor da chave privada
  correspondente à chave pública utilizada por \emph{B} poderá
  ler o texto \cite[p. 443]{SistemasOperacionaisModernos}. Esse
  método criptográfico de nada valeria para confirmar a
  identidade do autor de um documento se não fosse por uma
  característica do algorítimo RSA: o contrário também é
  verdadeiro. Um documento criptografado pela chave
  \emph{privada} de \emph{A}, conhecida apenas por este, só
  poderá ser descriptografado pela chave pública correspondente.
  Ou seja, se um determinado documento for legível após o uso da
  chave pública de \emph{A}, então é certo que foi criptografado
  por ele.\par

  O certificado digital e, paralelamente, a assinatura digital,
  utilizam conceito similar ao exposto. Como a chave pública de
  um indivíduo não tem restrições de divulgação, posto que não é
  possível descriptografar os documentos com ela criptografados,
  foram montadas estruturas para armazenar várias dessas chaves
  públicas em um único local, associando-as a determinadas
  pessoas. Um indivíduo poderia se cadastrar na empresa
  mantenedora dessa estrutura e armazenar lá a sua chave pública.
  Dessa forma, torna-se extremamente prático enviar arquivos
  criptografados para qualquer pessoa cadastrada em uma dessas
  empresas, que são chamadas de Autoridades Certificadoras. Ao
  efetuar esse cadastro junto às autoridades certificadoras, o
  indivíduo recebe um certificado digital, que contém sua chave
  privada, de conhecimento unicamente seu. A chave pública, em
  contra-partida, fica armazenada nos servidores da Autoridade
  Certificadora \cite[p. 16]{SegredoDemocracia}.

  Já a assinatura digital decorre de uma consequência indesejável
  da complexidade do algorítimo RSA: baixa velocidade
  \cite{ComputerNetworks}. A solução encontrada foi, ao invés de
  criptografar o documento por inteiro, utilizar-se de uma função
  de hash \textbf{(EXPLICAR HASH)}, que gera uma espécie de
  resumo criptográfico do arquivo. Apenas este resumo seria
  criptografado com a chave privada do remetente, garantindo
  assim velocidade no procedimento de assinatura do documento e
  garantindo um grau de confiabilidade
  aceitável \cite{ComputerNetworks}. Um documento que tenha
  passado por esse processo estaria assinado digitalmente pelo
  autor e não poderia ser alterado sem que fosse notado pelo
  receptor. Tendo em vista essa propriedade da assinatura
  digital, em 1996 foi aprovada pela Assembléia Geral da ONU a
  \emph{Lei Modelo do Comércio
  Eletrônico}, que estabeleceu requisitos para que o documento
  digital tenha função equivalente ao escrito
  \apud{LeiModeloComercioEletronico}{CertificadoDigital}.

	Segundo \citeonline{CertificadoDigital}, o documento digital é
  uma representação da realidade, seja textual gráfica ou sonora,
  que, normalmente, é volátil e de fácil alteração e
  falsificação. No entanto, através de mecanismos informáticos,
  como a já expicada certificação digital, é possível que
  adquiram as três características fundamentais de um documento
  em papel: a integridade, a autenticidade e a tempestividade, e
  assim possam servir às três funções dos documentos:
  identificativa, declaratória e probatória. \par
	
	No documento em papel, a integridade é verificada pela
  existência ou não de contrafação – rasura, escritos
  posteriores. No documento digital, a assinatura digital é
  inexoravelmente violada com qualquer mínima alteração no
  conteúdo do documento \cite{CertificadoDigital}.\par
	
	A identificação do autor pode ser realizada através de
  autoridades certificadoras, como a ICP-BRASIL, instituída pela
  Medida Provisória 2.200 \cite{MedProv2200}, que armazenam
  certificados para aferição de autenticidade das assinaturas
  digitais. \par
	
	Defendem \citeonline{CertificadoDigital} que a validade do
  documento eletrônico em si, apenas por ser eletrônico, não deve
  ser questionada.  Afinal, se a lei permite a existência de
  contratos verbais, por exemplo, não há porque negar validade
  aos contratos eletrônicos. Ainda segundo os autores, a
  volatilidade característica do meio eletrônico, que faz algumas
  pessoas manterem receio quanto a esses documentos é sanada pelo
  certificado digital. Justificam ainda o uso do documento
  eletrônico como meio de prova com fundamento no art. 322 do
  CPC, que determina serem todos os meios legais e moralmente
  legítimos hábeis para provar a verdade dos fatos. O documento
  digital só não poderia ser utilizado como prova para os casos
  em que a lei impõe determinada forma ao negócio jurídico que
  não é suportada pelos meios eletrônicos. \par
	
	Concluem \citeonline{CertificadoDigital} que a falta de
  regulamentação e de atribuição de validade jurídica aos
  documentos digitais representavam empecilhos ao desenvolvimento
  do comércio eletrônico. O entendimento mais correto é o de dar
  validade aos documentos digitais, apesar da relutância de
  alguns autores em admiti-lo, pois com a assinatura digital, o
  criador do documento tem a certeza de que o documento não
  será alterado sem o seu consentimento, assim como o
  destinatário poderá confiar na procedência e integridade da
  mensagem, assim como pode saber que foi enviado exatamente na
  hora indicada. \par

	
	Em relação à divulgação de atos do Poder Público via Internet,
  importante para o trabalho em virtude da existência de Diários
  da Justiça Eletrônicos e Intimações virtuais,
  \citeonline{PublicidadeINPI} discute-a em face do princípio da
  Publicidade Administrativa.  Segundo seu entendimento, a
  substituição feita pelo INPI – Instituto Nacional da
  Propriedade Intelectual – de sua revista impressa pela versão
  eletrônica fere o citado princípio por várias razões. Alega o
  autor que a Constituição Federal, ao instituir o princípio da
  Publicidade em seu art. 37 caput, determinou aos agentes
  públicos a adoção, de forma progressiva, dos comportamentos
  necessários à disseminação de seus atos tendente a atingir o
  maior grau possível de difusão e conhecimento por parte da
  população. \par
	
	Vedar a publicação impressa teria, portanto, afrontado esse
  mandado constitucional, convertendo-o, de um mandado de
  otimização, para, segundo \citeonline{PublicidadeINPI}, um
  mandado de pessimização, por restringir o alcance da publicação
  anteriormente existente.  Ainda segundo o entendimento do
  autor, com o péssimo grau de inclusão digital existente em
  nosso país, onde 79\% dos habitantes nunca manuseou um
  computador e 89\% nunca acessou a internet
  \apud{PesquisaInclusaoDigital}{PublicidadeINPI}, a supressão da
  da edição impressa da revista em nada contribui para a
  publicidade dos atos administrativos que passam a ser
  disponíveis apenas em forma eletrônica. \par
	
	
	Em seu artigo “A Informatização do Processo Judicial – da 'Lei
  do Fax' à Lei nº 11.419/2006: uma Breve Retrospectiva
  Legislativa”, o Dr. Demócrito Reinaldo Filho mostra a
  evolução do tratamento legislativo dado ao procedimento
  judicial virtual. Comentando a Lei 11.419/06, afirma que
  esta contemplou avanços tecnológicos recentes, englobando o
  Diário de Justiça online, a citação e intimação eletrônica,
  certificação digital, além de algumas alterações no CPC
  para acomodar as novas modificações.\par
	
	Segundo o autor, o debate acerca da idéia de um processo
  judicial completamente virtual teve início com um texto
  apresentado pela Ajufe à Comissão de Participação Legislativa
  da Câmara, em 2001. Este texto adotou o princípio da validade
  de todos os atos necessários à virtualização do processo: envio
  e recebimento de petições, armazenamento das peças, coleta de
  depoimentos e comunicação dos atos processuais.\par
	
	Afirma Reinaldo Filho que o primeiro passo na direção da
  virtualização foi dado pela Lei 9.800/99, que autorizou o uso
  de sistemas de transmissão de dados para a prática de atos
  processuais, incluindo-se aí o meio eletrônico. A Lei, no
  entanto, permitia apenas o trânsito eletrônico das peças
  processuais, devendo o seu armazenamento continuar a ser
  realizado em meio físico. A própria Lei não dispensou as partes
  de entregar os originais, além de autorizar apenas a
  transmissão de petições por meio eletrônico, não permitindo
  outros atos processuais, como audiências e sentenças.\par
	
	Em seguida, a Lei 10.259/01 disciplinou os Juizados Federais, e
  trouxe algumas inovações, como a autorização a recepção de
  peças processuais por meio eletrônico sem a exigência de
  posterior apresentação do original, ou a realização de sessões
  da Turma de Uniformização Jurisprudencial por videoconferência
  quando residentes em cidades diferentes. \par
	
	As implementações iniciais do sistema receberam críticas por
  não garantir a identidade dos usuários. Sem a inscrição
  presencial no sistema, qualquer um poderia se inscrever através
  do site alegando ser outra pessoa. Além dos Juizados Federais,
  os sistemas dos demais tribunais também não tinham
  implantado métodos de controle da identidade real do
  remetente.\par
	
	Em agosto de 2006, a Lei 11.341/06 permitiu aos recorrentes, em
  recurso especial ou extraordinário fundado em divergência
  jurisprudencial comprovar o dissídio com os julgados
  reproduzidos na Internet. No mesmo ano, a Lei 11.382/06 criou a
  penhora online, e o leilão online. \par
	
	Por fim, a Lei 11.419/06, marco inicial do processo legislativo
  de regulamentação do procedimento judicial eletrônico
  propriamente dito, admite o processo virtual em qualquer
  jurisdição e órgão. \par
