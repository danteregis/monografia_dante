 \chapter{Metodologia}

	O trabalho pretente utilizar o método hipotético-dedutivo para
  analisar a validade jurídica do processo judicial
  exclusivamente virtual. Pretende-se coletar dados acerca das
  implementações de sistemas de processo virtual nos tribunais
  com jurisdição em Sergipe e formular hipóteses, com base nos
  conceitos de documento eletrônico, assinatura digital e dos
  princípios processuais, a respeito da validade dos atos
  virtuais quando não assinados eletrônicamente.\par

  Com base nos pressupostos teóricos estabelecidos no capítulo
  \ref{marcoteorico}, pretende-se produzir um questionário a ser
  aplicado nos Tribunais com jurisdição no Estado: Tribunal de
  Justiça de Sergipe, Tribunal Regional Eleitoral, Tribunal
  Regional do Trabalho e Seccional da Justiça Federal. Através
  desse questionário e de entrevistas realizadas com funcionários
  dos departamentos de informática desses órgãos, pretende-se
  inferir se existe processo judicial virtual e se são utilizados
  métodos de certificação digital nos sistemas internos dos
  Tribunais. \par

  O questionário e a entrevista deverão coletar dados
  relacionados à tecnologia utilizada nos sistemas de processos
  virtuais, à necessidade ou não do uso de certificados
  digitais, ao processo necessário para obtenção dos
  certificados, ao uso de \emph{tokens} para autenticação por
  parte de juízes, serventuários da Justiça e advogados e às
  formas de garantia de integridade e segurança dos dados
  relacionados ao processo.\par

  De posse desses dados, o trabalho irá sintetizar o
  funcionamento desses sistemas e será realizada análise acerca
  da validade dos processos digitais em cada um dos
  Tribunais, culminando com a discussão relativa à validade
  jurídica do processo virtual digital em geral.\par
